\section{Examples\label{examples}}
\subsection{Feature demonstration\label{artificial}}
In this section some of the features of \sdd\ will be demonstrated using
artificially generated data, well known examples from other \RR\ functions
and the (slightly modified) examples of \sdd 's help file (cf.\ the Appendix).
The presentation starts with the latter, each example printed
on an individual page to obtain lucidity.

\vspace*{30mm}
\begin{center}\sl\small this space intentionally left blank
\end{center}

\clearpage\subsubsection{Helix}
In Figure \ref{helix} points of a helix are
calculated and plotted using the 3D highlighting mode (%
\verb|highlight.3d = TRUE|) in a blue box with a light blue grid.  We
produce the solid look with the point symbol, \texttt{pch = 20}.

\vspace*{10mm}
\begin{figure}[htb!]
\small
\begin{Verbatim}[frame=single]
  z <- seq(-10, 10, 0.01)
  x <- cos(z)
  y <- sin(z)
  scatterplot3d(x, y, z, highlight.3d = TRUE, col.axis = "blue",
                col.grid = "lightblue", main = "Helix", pch = 20)
\end{Verbatim}
\normalsize
\begin{center}
\includegraphics[width=13cm]{helix}
\end{center}
\vspace*{-10mm}\caption{Helix\label{helix}}
\end{figure}


\clearpage\subsubsection{Hemisphere}
Figure \ref{hemisphere} shows points on a hemisphere.
Except for  angle and the size of axes annotation, this figure is generated analogously to Figure \ref{helix}.

\vspace*{10mm}
\begin{figure}[htb!]
\small
\begin{Verbatim}[frame=single]
  temp <- seq(-pi, 0, length = 50)
  x <- c(rep(1, 50) %*% t(cos(temp)))
  y <- c(cos(temp)  %*% t(sin(temp)))
  z <- c(sin(temp)  %*% t(sin(temp)))
  scatterplot3d(x, y, z, highlight.3d = TRUE,  angle = 120,
           col.axis = "blue", col.grid = "lightblue", cex.axis = 1.3,
           cex.lab = 1.1, main = "Hemisphere", pch = 20)
\end{Verbatim}
\normalsize
\begin{center}
\includegraphics[width=13cm]{hemisphere}
\end{center}
\vspace*{-10mm}\caption{Hemisphere\label{hemisphere}}
\end{figure}


\clearpage\subsubsection{3D barplot}
With some simple modifications, it is possible to generate a 3D barplot, as shown in this example.
To make the plot look like a barplot, {\tt type = "h"} is set to draw vertical lines to the $x$--$y$ plane,
{\tt pch = " "} to avoid plotting of point symbols and {\tt lwd = 5} to make the lines looking like bars.
Furthermore, instead of three vectors a data frame is given as the first argument to \sdd .

\enlargethispage{1cm}
\vspace*{5mm}
\begin{figure}[htb!]
\small
\begin{Verbatim}[frame=single]
  my.mat <- matrix(runif(25), nrow = 5)
  dimnames(my.mat) <- list(LETTERS[1:5], letters[11:15])
  s3d.dat <- data.frame(columns = c(col(my.mat)),
                        rows    = c(row(my.mat)), value = c(my.mat))
  scatterplot3d(s3d.dat, type = "h", lwd = 5, pch = " ",
        x.ticklabs = colnames(my.mat), y.ticklabs = rownames(my.mat),
        color = grey(25:1 / 40), main = "3D barplot")
\end{Verbatim}
\normalsize
\begin{center}
\includegraphics[width=12.5cm]{barplot}
\end{center}
\vspace*{-10mm}\caption{3D barplot\label{barplot}}
\end{figure}


\clearpage\subsubsection{Adding elements}
The importance of \textsl{Lexical Scoping} to generate \textsl{function closures}
to provide extensibility of \sdd\ was discussed in Section \ref{extend}.
An example how to use the invisibly returned functions is given below on the famous
(at least for \textsf{S} users) tree data.

After the tree data is loaded, it is plotted by \sdd ,
and the (invisibly returned) result is assigned to the variable {\tt s3d}.
The (blue colored) points are plotted using {\tt type = "h"},
so one can see the $x$--$y$ location of those points very clearly.

In the next step, a linear model (assumption: volume depends on girth and height of the trees) is calculated.
Furthermore, this \textsf{lm} object is plotted by the returned {\tt plane3d} function (was assigned to {\tt s3d} before),
and it results in a regression plane.

Just for demonstration purposes, in the last step some red colored points (on a imaginary line crossing the plot)
are plotted with an asterisk as its point symbol.

\vspace*{10mm}
\small
\begin{Verbatim}[frame=single]
  data(trees)
  s3d <- scatterplot3d(trees, type = "h", color = "blue",
      angle = 55, scale.y = 0.7, pch = 16, main = "Adding elements")
  my.lm <- lm(trees$Volume ~ trees$Girth + trees$Height)
  s3d$plane3d(my.lm)
  s3d$points3d(seq(10, 20, 2), seq(85, 60, -5), seq(60, 10, -10),
      col = "red", type = "h", pch = 8)
\end{Verbatim}
%$
\normalsize

\clearpage
\begin{figure}[htb!]
\begin{center}
\includegraphics[width=13cm]{elements}
\end{center}
\vspace*{-10mm}\caption{Adding elements\label{elements}}
\end{figure}


\subsubsection{Bivariate normal distribution}
In Figure \ref{binorm} a surface of the density of a bivariate normal
distribution is plotted.  This example is a bit more sophisticated than the
examples before and shows the extensibility of \sdd.  Note that \sdd\
is designed to generate  scatter plots, not to draw surfaces,
is not really user friendly for this purpose, for which we'd typically
rather use \RR's \code{persp} function.

In a first step a matrix containing the density is calculated.
The call of \sdd\ sets up the plot (axes, labels, etc.), but doesn't draw
the surface itself which is accomplished by the two loops at the end of the
code.  Additionally, we give an example of quite sophisticated mathematical
annotation.
\clearpage
\small
\begin{Verbatim}[frame=single]
  library(mvtnorm)
  x1 <- x2 <- seq(-10, 10, length = 51)
  dens <- matrix(dmvnorm(expand.grid(x1, x2),
                    sigma = rbind(c(3, 2), c(2, 3))),
                 ncol = length(x1))
  s3d <- scatterplot3d(x1, x2,
                    seq(min(dens), max(dens), length = length(x1)),
                    type = "n", grid = FALSE, angle = 70,
                    zlab = expression(f(x[1], x[2])),
                    xlab = expression(x[1]), ylab = expression(x[2]),
                    main = "Bivariate normal distribution")
  text(s3d$xyz.convert(-1, 10, 0.07),
        labels = expression(f(x) == frac(1, sqrt((2 * pi)^n *
            phantom(".") * det(Sigma[X]))) * phantom(".") * exp{
            bgroup("(", - scriptstyle(frac(1, 2) * phantom(".")) *
                (x - mu)^T * Sigma[X]^-1 * (x - mu), ")")}))
  text(s3d$xyz.convert(1.5, 10, 0.05),
        labels = expression("with" * phantom("m") *
            mu == bgroup("(", atop(0, 0), ")") * phantom(".") * "," *
                phantom(0) *
            Sigma[X] ==  bgroup("(", atop(3 * phantom(0) * 2,
                2 * phantom(0) * 3), ")")))
  for(i in length(x1):1)
      s3d$points3d(rep(x1[i], length(x2)), x2, dens[i,], type = "l")
  for(i in length(x2):1)
      s3d$points3d(x1, rep(x2[i], length(x1)), dens[,i], type = "l")
\end{Verbatim}
\normalsize

\clearpage
\begin{figure}[htb!]
\begin{center}
\includegraphics[width=13cm]{binorm}
\end{center}
\vspace*{-12mm}\caption{Density of a bivariate normal distribution\label{binorm}}
\end{figure}

\subsubsection{RGB color cube}
In Figure \ref{colorcube}, we visualize the RGB (red--green--blue) color
space which \RR{} and most computer screens use for color coding.
First, we draw all the \emph{named} colors available in \RR{} via
\texttt{colors()}.  Note that it might be interesting to find a better
background color here than white. Optimally it would correspond to an RGB
location as far away as possible from all given colors.
%%\enlargethispage{1cm}% Trick, damit es passt ...
Second, we show the \texttt{rainbow()} colors in the RGB space.  Here we
redraw the points \emph{on top} of the cube, using the \texttt{points3d()}
closure which is also the basis of our \texttt{cubedraw()} function.
\begin{figure}[htb!]
\small
%% to save space, suppressed things like
%%       ## Purpose: Draw nice cube with corners

%%   par(mfrow = 1:2)

\begin{Verbatim}[frame=single]
  cubedraw <- function(res3d, min = 0, max = 255, cex = 2)
  {
      cube01 <- rbind(0,c(1,0,0),c(1,1,0),1,c(0,1,1),c(0,0,1),c(1,0,1),
                        c(1,0,0),c(1,0,1),1,c(1,1,0),
                        c(0,1,0),c(0,1,1),  c(0,1,0),0)
      cub <- min + (max-min)* cube01
      res3d$points3d(cub[ 1:11,], cex = cex, type = 'b', lty = 1)
      res3d$points3d(cub[11:15,], cex = cex, type = 'b', lty = 3)
  }
  crgb <- t(col2rgb(cc <- colors()))
  rr <- scatterplot3d(crgb, color = cc, box = FALSE, angle = 24)
  cubedraw(rr)

  Rrb <- t(col2rgb(rbc <- rainbow(201)))
  rR <- scatterplot3d(Rrb, color = rbc, box = FALSE, angle = 24)
  cubedraw(rR)
  rR$points3d(Rrb, col = rbc, pch = 16)
\end{Verbatim}
\par\vspace*{-12mm}%$
\normalsize
\centerline{\includegraphics[width=19cm]{colorcube}}
\par\vspace*{-5mm}
\caption{The RGB color cube. On the left, the named colors in \RR{}, i.e.,
  \texttt{colors()}. Note the diagonal of gray tones. On the right, the
  locations and colors of \texttt{rainbow(201)}.\label{colorcube}}
\end{figure}





\clearpage
\subsection{Real world examples\label{realworld}}
Three real world examples are presented in this section.
The data are from the following recent projects of the collaborative
research centre 475
(Deutsche Forschungsgemeinschaft, SFB 475:
 ``Reduction of complexity in multivariate data structures''):

\vspace{-5mm}\begin{description}
\item[C3 \textmd{(Biometrics)}] Meta--Analysis in Biometry and Epidemiology,
\item[B3 \textmd{(Econometrics)}] Multivariate Analysis of Business Cycles, and
\item[C5 \textmd{(Technometrics)}]
    Analysis and Modelling of the Deephole--Drilling--Process with Methods of Statistics and Neuronal Networks.
\end{description}


\subsubsection{Meta--analysis of controlled clinical trials\label{meta}}
In the first real world example the data from a project on ``Meta--Analysis
in Biometry and Epidemiology'' is taken.
The data set contains the results of 13 placebo--controlled clinical trials
which evaluated the efficacy of the Bacillus Calmette--Gu\'{e}rin (BCG)
vaccine for the prevention of tuberculosis (TB).
%
An important task in combining the results of clinical trials is to detect
possible sources of heterogeneity which may influence the true treatment
effect.
%
In the present example, a possible influential covariate is the distance of
each trial from the equator, which may serve as a surrogate for the
presence of environmental mycobacteria that provide a certain level of
natural immunity against TB.
%
Other covariates may be the year the trial was carried out and the
allocation scheme of the vaccination
(A = alternate, R = random, S = systematic).
%%
For more details, especially on the choice of the trials and the
meta--analytical methods of combining the results, we refer to
\citeN{knapp02} and the references given therein.

In Figure \ref{fig:meta} the estimated risks of TB disease are plotted for the
vaccinated group and the non--vaccinated group, respectively, in the
dependence of the year the trial was carried out, of the absolute distance
from the equator and of the allocation scheme.
The color represents the precisions of the estimated risks.
Figure \ref{fig:meta} clearly reveals a spatio--temporal trend in the realization of the trials.
The former trials were carried out far away from the equator,
and in all these trials one can observe an evident superiority of the BCG vaccine for the prevention of TB.
Except one trial all the other later trials were realized closer to the equator.
In these trials, it is apparently that the estimated risks in the non--vaccinated groups are even rather low and,
consequently, cannot graphically separated from the estimated risks in the vaccinated groups.
Finally, it is worthwhile to note that the later trial which was carried out far away from the equator
has a relative small estimated risk in the non--vaccinated group compared to the former trials and, hence,
does not yield such an evident superiority of the BCG vaccine.

\begin{figure}[b!]
\small
\begin{Verbatim}[frame=single]
  layout(cbind(1:2, 1:2), heights = c(7, 1))
  prc <- hsv((prc <- 0.7 * Prec / diff(range(Prec))) - min(prc) + 0.3)
  s3d <- scatterplot3d(Year, Latitude, Risk, mar = c(5, 3, 4, 3),
      type = "h", pch = " ", main = "Estimated TB risks")
  s3d$points(Year, Latitude, Risk, pch = ifelse(vac, 22, 21), bg = prc,
      cex = ifelse(vac, 2, 1.5))
  s3d.coords <- s3d$xyz.convert(Year, Latitude, Risk)
  al.char <- toupper(substr(as.character(Allocation), 1, 1))
  text(s3d.coords$x[!vac], s3d.coords$y[!vac], labels = al.char[!vac],
      pos = 2, offset = 0.5)
  legend(s3d$xyz.convert(80, 15, 0.21), pch = c("A", "R", "S"), yjust=0,
      legend = c("alternate", "random", "systematic"), cex = 1.1)
  legend(s3d$xyz.convert(47, 60, 0.24), pch = 22:21, yjust = 0,
      legend = c("vaccinated", "not vaccinated"), cex = 1.1)

  par(mar=c(5, 3, 0, 3))
  plot(seq(min(Prec), max(Prec), length = 100), rep(0, 100), pch = 15,
      axes = FALSE, xlab = "color code of variable \"Precision\"",
      ylab = "", col = hsv(seq(0.3, 1, length = 100)))
  axis(1, at = 4:7, labels = expression(10^4, 10^5, 10^6, 10^7))
\end{Verbatim}
\end{figure}
\normalsize

Three variables are represented by the three dimensions of the cube, while
variable ``Precision'' is represented by color.  To realize color
representation for metric variables, some manual tuning is necessary,
though.

Two kinds of point symbols stand for the ``Vaccinated'' variable,
and for a sixth variable, ``Allocation'', an appropriate letter is printed
additionally close to the ``not vaccinated'' symbol.

For each of the latter three variables a legend is desirable.
Thus the smaller two legends are plotted into the \sdd ,
while the legend for the color coding gets a single plot.
The function {\tt layout} arranges the two plots suitably on the same device.

\begin{figure}[htb!]
\begin{center}\includegraphics[width=13cm]{meta}\end{center}
\caption{Estimated TB risks\label{fig:meta}}
\end{figure}
\clearpage


\subsubsection{Business cycle data}
The example in this section shows the plotting of data from a project on ``Multivariate Analysis of Business Cycles''.
One of the main interests of the project is the prediction of business cycle phases.
An extraction of available relevant (concerning the purposes of this section) variables
and its abbreviations is given in Table \ref{StyFacts}.
The abbreviation 'gr' stands for growth rates with respect to last year's
corresponding quarter.

\begin{table}[htb!]
 \centering \vspace{0.2cm}
 \begin{tabular}{|l|l|}   \hline
 abbr & description                     \\ \hline
 IE & real investment in equipment (gr) \\
 C  & real private consumption (gr)     \\
 Y  & real gross national product (gr)  \\
 L  & wage and salary earners (gr)      \\ \hline
 \end{tabular}
 \caption{Abbreviations\label{StyFacts}}
\end{table}

The experts' classification of the data into business cycle phases
(``PH'') was done by \citeN{heilemann} using a 4-phase scheme.
These phases are called {\sl lower turning points}, {\sl upswing}, {\sl
  upper turning points}, and {\sl downswing}.

In Figure~\ref{business} the three variables C, Y, and L are represented by
the three dimensions of the cube.  The variable IE is represented by color,
while four different point symbols stand for the four business cycle phases
(PH).

For each of the latter two variables, a legend is desirable.
Thus the smaller one (for PH) is plotted into the \sdd , while the legend
for the color coding of IE got a single plot, analogously to the example in
Section~\ref{meta}.

A regression plane is added to the plot to support the visual impression.
Obviously all the plotted variables are highly correlated, with the
exception of the class variable which does not appear to be well
predictable by the other variables.  Details are discussed in
\citeN{theis99}.
%
In order to provide a correct impression of the fit, the residuals,
i.e. the projection lines to the plane, are drawn in Figure~\ref{residuals}
where different color and line types are used for positive and negative
residuals respectively.

\begin{figure}[H]
\vspace*{-10mm}
\footnotesize
\begin{Verbatim}[frame=single]
  layout(cbind(1:2, 1:2), heights = c(7, 1))
  temp <- hsv((temp <- 0.7 * IE / diff(range(IE))) - min(temp) + 0.3)
  s3d <- scatterplot3d(L, C, Y, pch = Phase, color = temp,
      mar = c(5, 3, 4, 3), main = "Business cycle phases")
  legend(s3d$xyz.convert(-2, 0, 16), pch = 1:4, yjust = 0,
      legend = c("upswing", "upper turning points",
          "downswing", "lower turning points"))
  s3d$plane3d(my.lm <- lm(Y ~ L + C), lty = "dotted")
  par(mar=c(5, 3, 0, 3))
  plot(seq(min(IE), max(IE), length = 100), rep(0, 100), pch = 15,
      axes = FALSE, xlab = "color code of variable \"IE\"", ylab = "",
      col = hsv(seq(0.3, 1, length = 100)))
  axis(1, at = seq(-20, 25, 5))
\end{Verbatim}
\normalsize
\begin{center}\includegraphics[width=13cm]{business}\end{center}
\vspace*{-5mm}\caption{Business cycle phases\label{business}}
\end{figure}

\begin{figure}[htb!]
\small
\begin{Verbatim}[frame=single]
  s3d <- scatterplot3d(L, C, Y, pch = 20, mar = c(5, 3, 4, 3),
            main = "Residuals")
  s3d$plane3d(my.lm, lty = "dotted")
  orig <- s3d$xyz.convert(L, C, Y)
  plane <- s3d$xyz.convert(L, C,  fitted(my.lm))
  i.negpos <- 1 + (resid(my.lm) > 0)
  segments(orig$x, orig$y, plane$x, plane$y,
            col = c("blue", "red")[i.negpos], lty = (2:1)[i.negpos])
\end{Verbatim}
%%$
\normalsize
\begin{center}\includegraphics[width=13cm]{residuals}\end{center}
\vspace*{-5mm}\caption{Residuals (cf. Figure \ref{business})\label{residuals}}
\end{figure}


\clearpage
\subsubsection{Deep hole drilling}
Our last real world example shows phase spaces (\cite{tong93}) of the drilling torque of a deep hole drilling process.
The data is taken from a project on
"Analysis and Modelling of the Deephole--Drilling--Process with Methods of Statistics and Neuronal Networks".
More detailed analysis on the data than provided in the following example was done by,
e.g., \citeN{busse} and \citeN{weinert}.

Figure \ref{drill1} visualizes the phase spaces of the drilling torques of two deep hole drilling processes,
a regular and a chattering one.
Obviously the points in the phase space of the chattering process are very systematically scattered,
and the range of the data is very different for the two processes.
The magnification of the regular process in Figure \ref{drill2} shows that the points of the regular process are scattered
unsystematically.
Note that other lags like 10, 20, 100 would produce a similar plot.
This indicates a sine wave like relationship in the chattering case.

\vspace{10mm}
\small
\begin{Verbatim}[frame=single]
  s3d <- scatterplot3d(drill1[1:400], drill1[7:406], drill1[32:431],
      color = "red", type = "l", angle = 120, xlab = "drilling torque",
      ylab = "drilling torque, lag 6", zlab = "drilling torque, lag 31",
      main = "Two deep hole drilling processes")
  s3d$points3d(drill2[1:400], drill2[7:406], drill2[32:431],
      col = "blue", type = "l")
  legend(s3d$xyz.convert(-400, 1000, 950), col= c("blue", "red"),
      legend = c("regular process", "chattering process"), lwd = 2,
      bg = "white")

  scatterplot3d(drill2[1:400], drill2[7:406], drill2[32:431],
      color = "blue", type = "l", angle = 120, xlab = "drilling torque",
      ylab = "drilling torque, lag 6", zlab = "drilling torque, lag 31",
      main = "Magnification of the regular process")
\end{Verbatim}
\normalsize

\begin{figure}[htb!]
\vspace*{-15mm}
\begin{center}\includegraphics[width=11.5cm]{drill1}
\vspace*{-10mm}\caption{Phase spaces of the drilling torques of two deep hole drilling processes\label{drill1}}
\vspace*{10mm}
\includegraphics[width=11.5cm]{drill2}\end{center}
\vspace*{-10mm}\caption{Magnification of the regular process (Figure \ref{drill1})\label{drill2}}
\end{figure}
\clearpage

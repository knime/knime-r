\HeaderA{pmml.rsf}{Generate PMML for a Random Survival Forest (rsf) object}{pmml.rsf}
\keyword{interface}{pmml.rsf}
\keyword{survival}{pmml.rsf}
\keyword{tree}{pmml.rsf}
\begin{Description}\relax
Generate the Predictive Model Markup Language (PMML) representation of
a \pkg{randomSurvivalForest} forest object.  In particular, this
function gives the user the ability to save the geometry of a forest
as a PMML XML document.
\end{Description}
\begin{Usage}
\begin{verbatim}
## S3 method for class 'rsf':
pmml(model, model.name="rsfForest_Model", app.name="Rattle/PMML",
     description="Random Survival Forest Tree Model", copyright, ...)
\end{verbatim}
\end{Usage}
\begin{Arguments}
\begin{ldescription}
\item[\code{model}] the \code{forest} object contained in an object of class
\pkg{randomSurvivalForest}, as that contained in the object
returned by the function \code{rsf} with the parameter
\dQuote{forest=TRUE}.
\item[\code{model.name}] a name to give to the model in the PMML.
\item[\code{app.name}] the name of the application that generated the PMML.
\item[\code{description}] a descriptive text for the header of the PMML.
\item[\code{copyright}] the copyright notice for the model.
\item[\code{...}] further arguments passed to or from other methods.
\end{ldescription}
\end{Arguments}
\begin{Details}\relax
The Predictive Model Markup Language is an XML based language which
provides a way for applications to define statistical and data mining
models and to share models between PMML compliant applications.  More
information about PMML and the Data Mining Group can be found at
http://www.dmg.org.  

Use of PMML and \code{pmml.rsf} requires the \pkg{XML} package.  Be
aware that XML is a very verbose data format.  Reasonably sized trees
and data sets can lead to extremely large text files.  XML, while
achieving interoperability, is not an efficient data storage mechanism
in this case.

This function is used to export the geometry of the forest to other
PMML compliant applications, including graphics packages that are
capable of printing binary trees.  In addition, the user may wish to
save the geometry of the forest for later retrieval and prediction on
new data sets using \code{pmml.rsf} together with \code{pmml\_to\_rsf}.
\end{Details}
\begin{Value}
An object of class \code{XMLNode} as that defined by the \pkg{XML}
package.  This represents the top level, or root node, of the XML
document and is of type PMML. It can be written to file with
\code{saveXML}.
\end{Value}
\begin{Note}\relax
One cautionary note is in order.  The PMML representation of the
\pkg{randomSurvivalForest} forest object is incomplete, in that the
object needs to be massaged in order for prediction to be possible.
This will be clear in the examples.  This deficiency will be addressed
in future releases of this package.  However, it was felt that the
current functionality was important enough and mature enough to
warrant release in this version of the product.
\end{Note}
\begin{Author}\relax
Hemant Ishwaran \email{hemant.ishwaran@gmail.com} and Udaya B. Kogalur
\email{ubk2101@columbia.edu} with incorporation into the pmml package
by \email{Graham.Williams@togaware.com}
\end{Author}
\begin{References}\relax
H. Ishwaran and Udaya B. Kogalur (2006).  Random Survival Forests.
\emph{Cleveland Clinic Technical Report}.

PMML home page: \url{http://www.dmg.org}
\end{References}
\begin{SeeAlso}\relax
\code{\LinkA{pmml}{pmml}}.
\end{SeeAlso}
\begin{Examples}
\begin{ExampleCode}
library(randomSurvivalForest)
data(veteran, package = "randomSurvivalForest")
veteran.out <- rsf(Survrsf(time, status)~.,
        data = veteran,
        ntree = 5,
        forest = TRUE)
veteran.forest <- veteran.out$forest
pmml(veteran.forest)
\end{ExampleCode}
\end{Examples}


\HeaderA{pmml.ksvm}{Generate PMML for a ksvm object}{pmml.ksvm}
\keyword{interface}{pmml.ksvm}
\begin{Description}\relax
Generate the PMML representation for a ksvm object (SVM). The PMML can
then be imported into other systems that accept PMML.  With this code,
a PMML representation can be obtained for SVMs implementing
classification (multi-class and binary) as well as regression.
\end{Description}
\begin{Usage}
\begin{verbatim}
## S3 method for class 'ksvm':
pmml(model, model.name="SVM_model", app.name="Rattle/PMML",
     description="Support Vector Machine PMML Model", copyright=NULL,
     data.name, ...)
\end{verbatim}
\end{Usage}
\begin{Arguments}
\begin{ldescription}
\item[\code{model}] a ksvm object.
\item[\code{data.name}] the name of the data object used to train the SVM
model in ksvm - required since the ksvm object does not appear to
record information about the used categorical variables.
\item[\code{model.name}] a name to give to the model in the PMML.
\item[\code{app.name}] the name of the application that generated the PMML.
\item[\code{description}] a descriptive text for the header of the PMML.
\item[\code{copyright}] the copyright notice for the model.
\item[\code{...}] further arguments passed to or from other methods.
\end{ldescription}
\end{Arguments}
\begin{Details}\relax
The generated PMML can be imported into any PMML consuming application
that recognizes PMML 3.2. An example is ADAPA (Adaptive Decision and
Predictive Analytics), a lightweight decision engine with batch and
real-time scoring of predictive models (also supporting neural
networks and linear and logistic regression).
\end{Details}
\begin{Author}\relax
\email{info@zementis.com}
\end{Author}
\begin{References}\relax
Package home page: \url{http://rattle.togaware.com}

PMML home page: \url{http://www.dmg.org}

Zementis' useful PMML convert: \url{http://www.zementis.com/pmml_converters.htm}

ADAPA home page: \url{http://www.zementis.com/adapa.htm}
\end{References}
\begin{SeeAlso}\relax
\code{\LinkA{pmml}{pmml}}.
\code{\LinkA{ksvm}{ksvm}}.
\end{SeeAlso}
\begin{Examples}
\begin{ExampleCode}
# Train a support vector machine to perform binary classification.
require(kernlab)
data(spam)
index <- sample(1:dim(spam)[1])
ds <- spam[index[1:300],] # For illustration only use a small dataset.
fit <- ksvm(type~., data=ds, kenrel="rbfdot")

# Genetate the PMML.
pmml(fit, data=ds)
\end{ExampleCode}
\end{Examples}

